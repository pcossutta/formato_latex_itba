% !TEX TS-program = LuaLaTeX
% !TEX encoding = UTF-8 Unicode
\documentclass[10pt]{book} 
\usepackage[T1]{fontenc}			% Output font encoding for international characters
\usepackage{fontspec}
\setmainfont{avenirnext.ttc}[
  UprightFeatures    = {FontIndex=7} ,
  BoldFeatures       = {FontIndex=5} ,
  ItalicFeatures     = {FontIndex=4},
  BoldItalicFeatures = {FontIndex=6} ,
]
% 0 bold - 1 bold it - 2 semi bold - 3 semi bold it - 4 it - 5 medium - 6 medium it - 7 normal - 8 heavy - 9 heavy italic - 10 ultralight - 11 ultralight it
\usepackage{graphicx}
%\graphicspath{ {figs/} }
\usepackage[onehalfspacing]{setspace}
\usepackage[a4paper, width=154mm, top=33mm, bottom=33mm, bindingoffset=6mm]{geometry}
\RequirePackage{caption} 		% Required for customising the captions
\captionsetup{justification=centerlast,font=small,labelfont=sc,margin=1cm}
\usepackage{hyperref}
\hypersetup{
    colorlinks=true,
    linkcolor=blue,
    filecolor=magenta,      
    urlcolor=blue,
    citecolor=blue,    
}
\usepackage[spanish, es-tabla, es-nodecimaldot]{babel}
%\usepackage{fancyhdr}				% Falta configurarlo bien
\usepackage[square, numbers]{natbib}	% Bibliography
\usepackage{tikz}					% Para poder dibujar el recuadro de la página del título
\usetikzlibrary{babel}					% tikz no funciona con babel si no se indica
% Agregados por el usuario
\usepackage{amssymb,amsmath}
\usepackage[only,Yup]{stmaryrd} 		% El símbolo de la estrella
\usepackage{bibentry} 				% Para poner las referencias en la intro
\usepackage[section]{placeins}			% Para que termine de colocar las figuras cuando finaliza un capítulo
\nobibliography*					% Sino no anda la bibliografía

\frontmatter
\begin{document}
%
\begin{titlepage}
	% Si algo hay que sacar porque no existe, ponerlo entre \phantom{...} 
	% para que respete la ubicación de las cosas
	\begin{tikzpicture}[remember picture, overlay]
		\coordinate (top_right) at 
		    ([xshift=-2.5cm, yshift=-2.5cm]current page.north east);
		\coordinate (top_left) at 
		    ([xshift=1.6cm, yshift=-1.6cm]current page.north west);
		\coordinate (bottom_right) at 
		    ([xshift=-1.6cm, yshift=1.6cm]current page.south east);
		\node[inner sep=0, anchor=north east] at (top_right) {\includegraphics[height=19mm, trim={180 200 200 200}, clip]{figs/logo_itba.png}};
		\draw[double, line width = 0.5pt] (top_left) rectangle (bottom_right);
	\end{tikzpicture}
	\onehalfspacing
	\par
	\begin{large}
		\noindent \href{http://www.itba.edu.ar}{\color{black}{\textbf{INSTITUTO TECNOLÓGICO DE BUENOS AIRES}}}\par
		\noindent \textbf{DEPARTAMENTO DE DOCTORADO}\par
		\vspace{4cm}
		\begin{center}
			{\Huge \textbf{TÍTULO DEL TRABAJO}\par}
			{\Huge \textbf{Subtítulo del trabajo (si lo hubiere)}\par}
		\end{center}
		\vspace{3.2cm}
		\noindent \textbf{AUTOR:} Ing. Nombre \textsc{Apellido}\par
		\vspace{0.5cm}
		\noindent \textbf{DIRECTOR:} Dr./Dra. Nombre \textsc{Apellido}\par
		\noindent \textbf{CO-DIRECTOR:} Dr./Dra. Nombre \textsc{Apellido}\par	
		\vspace{3cm}
		\noindent{TESIS PRESENTADA PARA LA OBTENCIÓN DEL TÍTULO DE}\par
		\noindent\textbf{DOCTOR EN INGENIERÍA / informática}\par
		\vfill
		\begin{center}
			\textbf{BUENOS AIRES}\\
			\textbf{PRIMER / SEGUNDO CUATRIMESTRE, 2018}
		\end{center}
	\end{large}
\end{titlepage}

%\maketitle
\tableofcontents 
\listoffigures			% Listado de Figuras
\listoftables			% Listado de Tablas

\chapter*{Abreviaturas}
\addcontentsline{toc}{chapter}{Abreviaturas}
\begin{table}[htb] 		% Si se pasa de una hoja usar longtables
	\large
	\begin{tabular}{l l}
		ITBA & \textbf{I}nstituto \textbf{T}ecnológico de \textbf{B}uenos \textbf{A}ires \\
		MIT & \textbf{M}assachusetts \textbf{I}nstitute of \textbf{T}echnology, Instituto Tecnológico de Massachusetts \\
		Cap. & \textbf{C}apítulo
	\end{tabular}
\end{table}

\mainmatter
%\pagestyle{fancy}
\pagenumbering{arabic}
% !TEX encoding = UTF-8 Unicode
% !TEX root = ../thesis.tex
\chapter{Introducción} \label{intro}
Este es un capítulo de ejemplo.
\section{Configuración de Página}
Los siguientes son los parámetros de configuración de la página de este formato en latex.
\pagevalues

Esto que sigue es un footnote: %\footnote{footnote number 1 working fine}
%
\begin{figure}
	\pagediagram
	\caption{Diagrama de página}
	\label{fig1_intro}
\end{figure}

La Fig. \ref{fig1_intro} muestra el formato que adopta Latex para el Layout.

Como se puede apreciar en la Fig. \ref{fig2_intro}, los lápices tienen muchos colores, los mismos se pueden generar de diversas formas \cite{ejemplo}.

Las referencias pueden ser en formato APA o IEEE.

\begin{figure}[htb]
	\centering
	\includegraphics[width=8.5cm]{figs/chapter1/sample.jpg}
	% El que está entre corchetes va al Toc, el otro al documento
	\caption[Figura de ejemplo]{Figura de ejemplo, se incluye un texto lo suficientemente largo como para ver como queda distribuido en el documento}
	\label{fig2_intro}
\end{figure}

\marginpar{Esto es una nota al margen, lo que sigue es Lipsum}
\lipsum[15-18]

En la Tabla \ref{table1_intro} se observan los colores disponibles.
\begin{table}[htb]
\renewcommand{\arraystretch}{1.3}
	\caption{Tabla de Colores}
	\label{table1_intro}
	\centering
	\setlength\tabcolsep{2pt}
	\begin{tabular}{c c}
		\hline
		\bfseries Código & \bfseries Color\\
		\hline
		$\mathbf{v_1}$ & Rojo\\
		$\mathbf{v_2}$ & Azul\\
		$\mathbf{v_3}$ & Verde\\
		\hline
	\end{tabular}
\end{table}

Dada la ecuación de ejemplo:
\begin{equation}
	\label{eq1_chapter1}
	E=m\,c^2
\end{equation}
Si reemplazamos $m$ en \eqref{eq1_chapter1} obtenemos la energía.

% Se agrega acá para no agregar paquetes extra al formato.
\section{Lorem Ipsum}
\lipsum[1-15]
Esto que sigue es otro footnote: \footnote{footnote number 2 working fine}
%% !TEX encoding = UTF-8 Unicode
% !TEX root = ../thesis.tex
\chapter{Manejo Formato en \LaTeX } \label{LatexFiles}

En este capítulo se dan instrucciones generales para el manejo del formato en \LaTeX. El formato consta de diferentes archivos, que serán detallados a continuación. 

\section{Lista de Archivos}

Los archivos \textit{thesis.tex}, \textit{copyright.tex}, \textit{engcopyright.tex} y \textit{EnglishCover.tex} corresponden a la configuración y estructura del formato del documento, y no deberían ser modificados por el usuario. En la Tabla~\ref{tab:UserFiles} se presentan la ubicación, nombre y contenido de cada archivo modificable por el usuario, donde será incluido el contenido de la tesis.

\begin{table}[ht]
\renewcommand{\arraystretch}{1.3}
    \centering
    \begin{tabular}{>{\centering}p{0.13\textwidth}p{0.16\textwidth}p{0.65\textwidth}} \hline  
    \textbf{Carpeta}  & \textbf{Archivo} & \textbf{Contenido} \\ \hline 
        & \textit{bibliography.bib} & Archivo de base de datos de las referencias bibliográficas.  \\ \hline
    \multirow{6}{*}{preamble} & \textit{abstract } &  Resumen en inglés de la tesis \\ 
     & \textit{dedicatory} & Dedicatoria \\  
     & \textit{resumen} & Resumen en castellano de la tesis \\  
     & \textit{thanks} & Agradecimientos \\  
     & \textit{title\_data} & Información para la carátula de la tesis: título, autor(a), director(a), codirector(a) y jurado \\ 
     & \textit{user\_packages} & Incluir paquetes específicos requeridos por el usuario. Los paquetes \textit{graphix, array, longtable} y \textit{amsmath} ya están incluidos en el formato. \\ \hline
     \multirow{2}{*}{chapters} & \textit{chapter1} &  Contenido del capítulo \\ 
    & \textit{chapters\_list} & Lista de capítulos a incluir en el documento. \\ \hline
    \multirow{2}{*}{appendices} & \textit{appendices\_list} &  Lista de apéndices a incluir en el documento. \\ 
    & \textit{appendix1} & Contenido del apéndice \\ \hline
     figs & \textit{chapter1} &  Carpeta para incluir las imágenes para el capítulo 1 \\ \hline
    \end{tabular}
    \caption{Archivos modificables por el autor}
    \label{tab:UserFiles}
\end{table}

Para quitar la portada en inglés en \LaTeX, desactive la línea \verb!\input{preamble/EnglishCover.tex}! en el archivo \textit{thesis.tex}

Dentro del documento, las referencias se utilizarán con estilo APA o IEEE. Para elegir el estilo, al final del archivo \textit{thesis.tex}, modifique la línea \verb!\bibliographystyle{ }! con \verb!ieeetr! para el estilo IEEE, o \verb!apalike! para APA.
%\include{chapters/chapter3}

\appendix % Los próximos "chapters" son Apéndices
% !TEX encoding = UTF-8 Unicode
% !TEX root = ../thesis.tex
\chapter{Título del Apéndice} \label{app1}
Este es un apéndice.

\textit{Esto es texto en itálica}.

\textbf{Esto es texto en negrita}.

\textsc{Esto es texto en Small Caps}

\textit{\textbf{Esto es texto en negrita itálica}}.

%\include{appendices/appendix2}

\bibliographystyle{ieeetr}
\bibliography{bibliography}
\addcontentsline{toc}{chapter}{\bibname}    %Agrega bibliografía al índice

\end{document}  
